%\documentclass[12pt,letter]{report}
\documentclass[12pt,oneside,openany,letter]{book}
\usepackage[colorlinks=true,bookmarksnumbered,linktocpage,pdftex]{hyperref}
\usepackage{hyperref}
\usepackage[activeacute,spanish]{babel}
\usepackage[utf8]{inputenc}
\usepackage{amsmath}
\usepackage{float}
\usepackage{epsfig}
\usepackage{graphicx}
\usepackage{titlesec}
\usepackage{multirow}
\usepackage{graphicx}
\usepackage{caption}
\usepackage{subcaption}
\usepackage{color}
\usepackage{calc}
\usepackage[nottoc]{tocbibind}
\usepackage{titletoc}
\usepackage{capt-of}
\usepackage{verbatim}
\usepackage[titletoc,title]{appendix}
\usepackage[left=3cm, right=3cm,top=3cm, bottom=2.5cm]{geometry}
\usepackage[table,xcdraw]{xcolor}

\newcommand{\bigrule}{\titlerule[1mm]}
\definecolor{c1}{rgb}{0,0.5,0}
\definecolor{c2}{rgb}{0.9,.0,0}
\definecolor{c3}{rgb}{.465,.535,.605}%color del panel
\definecolor{c4}{rgb}{.6,.6,.6}%color de los botones del panel

\hypersetup{linkcolor=blue}                         
\hypersetup{citecolor=red}

%%%%FORMA DE LOS CAPITULOS SECCIONES...ETC%%%%%%

%========================================
% Formato de capítulos y secciones
%\newcommand{\esp}{\rule{0in}{3ex}} %para crear espacios en las tablas
\usepackage{titlesec}
\titleformat{\section}[hang]{\bfseries} {\Large\thesection}{12pt}{\Large}[{\titlerule[0pt]}]
 % Para hacer una línea debajo de cada sección
\titleformat{\chapter}[display] % cambiamos el formato de los capítulos
{\bfseries\Huge} % por defecto se usarán caracteres de tamaño \Huge en negrita
{ \filleft % texto alineado a la derecha
 \Large\chaptertitlename\ % "Capítulo" o "Apéndice" en tamaño \Large en lugar de \Huge
 \Large\thechapter} % número de capítulo en tamaño \Large
{0mm} % espacio mínimo entre etiqueta y cuerpo
{\filleft} % texto del cuerpo alineado a la derecha
[\vspace{0.5mm} \bigrule] % después del cuerpo, dejar espacio vertical y trazar línea horizontal gruesa



\setlength{\parskip}{10pt} %espacio entre párrafos
\makeatletter
\newcommand\figcaption{\def\@captype{figure}\caption}
\makeatother


%%%%%%%%%% NEW COMANDS %%%%%%%%%%%%%%%%%%%%%%%%%%%%%%%%%%%%%%%%%55

%%%%%%%%%%%%%%%%%%%%%%%%%%%%%%%%%%%%%%%%%%%%%%%%%%%%%%%%%%%%%%%%%%%%%%%%%%%%%%
\parindent0cm %Sangria
\parskip0.5cm %Espacio entre prrafos.
\baselineskip0.5cm
\begin{document}
\begin{titlepage}


\centering {\Large {\sc  ELVES: Fenómenos Luminosos Transitorios de la alta Atmósfera}}

\vfill

\centering {\Large \textbf{Adriana Carolina Vásquez Ramírez} $^{1}$}

\vfill

\centering {\Large \textbf{Director:} Luis A. Núñez $^{1}$}

\centering {\Large \textbf{Co-Director:} Enrico Arnone $^{2}$}

\centering {\Large \textbf{Co-Director:} Roberto Mussa $^{3}$}

\vfill

$^{1}$  Escuela de F\'isica, Universidad Industrial de Santander \\
$^{2}$ Departamento de F\'isica, Universidad de Tur\'in
\\
$^{3}$ Instituto Nacional de F\'isica Nuclear,  Secci\'on de Tur\'in\\
\vfill

\centering {\Large Universidad Industrial de Santander\\Facultad de
Ciencias\\Escuela de F\'{i}sica\\2020}
\end{titlepage}

\tableofcontents
\listoffigures
%\listoftables
%\newpage


%%%%%%%%%%%%%%%%%%%%%%
%%%%% CHAPTER 1  %%%%%
%%%%%%%%%%%%%%%%%%%%%%
\chapter*{Introducci\'on}\label{introduccion}
\addcontentsline{toc}{chapter}{Introducci\'on}
Desde el siglo pasado se empezaron a reportar fen\'omenos luminosos ascendentes que se produc\'ian por encima de las tormentas el\'ectricas, por lo que se denominaban rayos de nube a la alta atmósfera, nube al espacio o nube a la ionosfera. M\'as adelante, el profesor Winckler \cite{FranzEtal1990} captur\'o con una c\'amara de v\'ideo de alta sensibilidad, las primeras im\'agenes de lo que hoy se conoce como un Sprite. El v\'ideo revel\'o un rayo ascendente de columnas gemelas de luz, con longitudes de decenas de kil\'ometros, lo que llam\'o la atenci\'on del programa de vuelos espaciales de Estados Unidos, que ya hab\'ia tenido encuentros desafortunados con estos rayos. Fue as\'i como surgieron varios programas para el estudio de los Eventos Luminosos Transitorios, denotados como TLEs. 

Los TLEs son emisiones cortas de luz que se originan sobre una tormenta a partir de las decargas el\'ectricas. Hoy en d\'ia se conocen varios tipos: Sprites, Blue Jets, Jets Gigantes, ELVES y Halos. Los Sprites consisten de una cascada de plasma electrificado con ramas de color rojizo y azul, que se propagan hacia abajo y luego con una expansi\'on en luminosidad hacia arriba. Los Blue Jets son rayos que salen del tope de la nube hacia arriba alcanzando extensiones de 30 a 40 km. Por otra parte, los ELVES y los Halos son emisiones de luz en forma de disco cuyo di\'ametro puede extenderse hasta unos 300 km para los ELVES y unos 100 km para los Halos. 

Los ELVES se generan en la ionosfera, por encima de los otros TLEs, por el pulso electromagnético de una descarga. Este pulso se propaga como una onda esférica desde la base del canal del rayo, intersectando la ionosfera como un anillo que se expande, produciendo la calefacci\'on de electrones libres y aumentando las excitaciones inducidas por colisi\'on, la ionizaci\'on y las emisiones de luz \'optica. Los ELVES ocurren con m\'as frecuencia que los dem\'as TLEs. Adem\'as, a diferencia de los Sprites y los Blue Jets, los ELVES son imperceptibles al ojo humano debido a su corta duraci\'on de decenas de microsegundos. Por lo general se detectan con c\'amaras de alta velocidad, arreglos de fot\'ometros o fotomultiplicadores, con resoluciones temporales del orden de los $\mu$s. Existen varios programas como el ISUAL, TARANIS, Firefly y el ASIM diseñados para la observación de los ELVES desde la atm\'osfera. Por otro lado, el Observatorio Pierre Auger, diseñado para el estudio de rayos Cósmicos de ultra alta energía (ubicado en Malargue, Argentina), detectó por accidente un ELVES, por lo que implementó un programa de Cosmo-geofísica para el estudio de estos fenómenos. 

La dinámica de los ELVES aún no se entiende por completo y todavía hay muchos temas en discusión, por ejemplo, se han detectado rayos gamma terrestres sobre las tormentas que podrían estar asociados a los ELVES. En este reporte se describe la estructura de las tormentas eléctricas y los tipos de descarga en el capítulo \ref{tormentas}, las características de los fenómenos en la alta atmósfera en el capítulo \ref{fenomenos} y la detección global de los ELVES en el capítulo \ref{deteccion}. 

%%%%%%%%%%%%%%%%%%%%%%%%%%%%%%%%%%%%%%%%%%%%%%%
\chapter{Tormentas eléctricas}\label{tormentas}
Las tormentas eléctricas se producen en nubes densas de gran extensión vertical, conocidas como cumulonimbos, que se forman a partir del vapor de agua transportado por potentes corrientes de aire ascendente. Este fenómeno meteorológico se distingue por la presencia de rayos y truenos, que usualmente van acompañados de vientos y lluvias fuertes, o a veces de granizo o de nieve. 

Por encima de estas tormentas, a diferentes altitudes de la atmósfera, ocurren los denominados Eventos Luminosos Transitorios. Para explicar el origen de estos fenómenos es necesario conocer la dinámica de las tormentas, así como sus características y distribución global. En las secciones siguientes se describe la estructura de carga más simple de las tormentas eléctricas, los lugares donde se originan los rayos, su fenomenología, terminología y propagación en la atmósfera. También se muestra la distribución global y las redes de monitoreo de rayos alrededor del mundo.  

\section{Estructura de carga de las tormentas}
\begin{figure}
    \centering
    \includegraphics[scale=0.62]{figures/tormenta_modelo.png}
    \caption[Estructura de carga según el modelo de tripolo.]{Estructura de carga de dos tormentas simples aisladas. Según el modelo de tripolo la parte superior de la nube está dominada por la carga positiva principal, seguida de la capa negativa principal y una pequeña capa de carga positiva en la parte más baja. Además se muestran algunos sitios donde se pueden producir los rayos: dentro de la nube (IC), de nube a tierra (CG), entre dos nubes (CC) e incluso de nube al aire (CA). Tomado de \cite{DwyerUman2014}.}
    \label{fig:tormenta_modelo}
\end{figure}

La estructura de carga ideal de las tormentas eléctricas está definida por el modelo tripolar estándar, donde la nube tiene un centro de carga negativa principal, uno de carga positiva principal y un centro de carga positiva inferior (ver figura \ref{fig:tormenta_modelo}). Frecuentemente aparece una cuarta región de carga negativa significativa, llamada capa superior de apantallamiento, que expulsa las líneas de campo eléctrico producidas por las cargas de la tormenta eléctrica. Esta capa se forma debido a la mayor conductividad del aire limpio fuera de la nube, especialmente en la estratosfera sobre la tormenta \cite{DwyerUman2014}. 

El proceso de transferencia de la carga primaria en la nube, ocurre a través de las colisiones entre partículas de granizo blando y de pequeños cristales de hielo. El granizo es lo suficientemente pesado como para caer o permanecer inmóvil en las corrientes de aire ascendente de la tormenta, mientras que los cristales de hielo son lo suficientemente ligeros para dejarse llevar hacia arriba por esas corrientes \cite{DwyerUman2014}. Generalmente, el centro de carga negativa principal se encuentra en un rango de temperatura entre -10 y -25 $^{\circ}$C, independientemente de la altura del terreno debajo de la tormenta \cite{RakovEtal2003}. Estos ingredientes son necesarios para que opere el mecanismo de carga no inductivo \cite{MacgormanEtal1998}. Por ejemplo, para las tormentas de verano en Florida y Nuevo México, estas temperaturas ocurren aproximadamente entre 6 y 8 km s.n.m \cite{DwyerUman2014}, mientras que en el caso de las tormentas invernales japonesas estas temperaturas se producen a altitudes mucho más bajas (alrededor de los 2 km s.n.m. \cite{KrehbielEtal1986}), debido a que la atmósfera allí es más fría. 

Por otra parte, la región de carga positiva principal suele ser más difusa que la capa negativa principal y reside en la zona superior de la nube. Dependiendo de la altura de la tormenta, la carga positiva superior puede oscilar entre 8 y 15 km de altitud en las tormentas de verano, mientras que en invierno su grosor es de unos pocos km. La región positiva inferior se encuentra debajo de la capa principal negativa y el fondo de la nube visible; por ejemplo, en las tormentas de verano en Florida estaría por encima de los 2 km de altitud \cite{DwyerUman2014}. 

La estructura de carga en una tormenta eléctrica es en realidad más compleja que la que se muestra en la figura \ref{fig:tormenta_modelo}, puede variar de una tormenta a otra y en ocasiones es muy diferente de lo ilustrado. Esta estructura también puede evolucionar en el tiempo, debido a que los rayos pueden depositar y reorganizar la carga dentro de la tormenta. En conjunto, la estructura de carga y los campos eléctricos de las tormentas son complejos, dependiendo del tiempo, la ubicación y el tipo de tormenta, por lo que se debe tener cuidado al interpretar las observaciones a partir de un modelo específico \cite{DwyerUman2014}. Además, las dos nubes aisladas de la figura \ref{fig:tormenta_modelo} podrían ser parte de varias tormentas contiguas e interactivas que comprenden sistemas de tormentas más grandes y complejos. 



%%%%%%%%%%%%%%%%%%%%%%%%%%%
\section{Fenomenología de los rayos}\label{fenomenologia}
Los rayos se pueden definir como una chispa eléctrica muy larga que generalmente ronda entre 5 y 10 km de longitud, con algunos casos extremos de 100 km \cite{DwyerUman2014}. Las descargas eléctricas en las tormentas pueden durar alrededor de los 0.5 s y producen pulsos electromagnéticos que se propagan en la atmósfera. Los rayos son los fenómenos más impresionantes y comunes en Geofísica, pues provocan la luz más brillante (relámpago) y el sonido más fuerte (trueno) en la Tierra, aunque su aparición aparentemente aleatoria en el espacio y el tiempo, la amplia gama de su variación temporal y el oscurecimiento de la nube que lo produce, hace que sean particularmente difíciles de estudiar.

\begin{figure}
    \centering
    \includegraphics[scale=0.5]{cg_development.png}
    \caption[Desarrollo de una descarga negativa de nube a tierra]{Esquema del desarrollo de una descarga negativa de nube a tierra. El l\'ider escalonado se va propagando hacia la tierra induciendo una carga positiva sobre esta superficie. Una de estas descargas ascendentes entra en contacto con una rama del líder, determinando el canal principal de la corriente entre la nube y la tierra. A partir del impacto de retorno, la corriente se propaga continuamente hacia arriba con un tercio de la velocidad de la luz. Después de unos 40 ms se pueden producir otros impactos en el canal principal \cite{DwyerUman2014}.}
    \label{fig:cg_development}
\end{figure}

Tomando en cuenta la estructura ideal de carga de las nubes, ilustrada en la figura \ref{fig:tormenta_modelo}, se puede notar que no existe una sola ubicación para la iniciación de los rayos, puesto que no hay una estructura única de carga para las tormentas eléctricas. Sin embargo, se han observado ciertos patrones que ayudan a clasificar las descargas eléctricas en dos categorías \cite{DwyerUman2014}: las que unen la brecha entre la carga de la nube y la tierra, y las que no. Las primeras se conocen como descargas de nube a tierra, denotadas como CG. Las segundas se dividen dependiendo del lugar donde se producen: dentro de la nube (denotada como IC), entre dos nubes (CC) y de la nube al aire (CA). Este grupo representa la mayoría de todas las descargas de rayos, siendo las IC las más comunes de todas las formas, seguidas de las CC y de las CA. 

Adem\'as, las descargas CG se clasifican en cuatro tipos dependiendo de su desarrollo, es decir, del signo de la carga eléctrica y la dirección de propagación del l\'ider inicial. La m\'as com\'un, denotada como -CG, es la que lleva carga negativa y se conecta con la carga positiva de la tierra. Un sistema típico de tormenta eléctrica pequeña produce un relámpago CG cada 20 a 30 s durante 40 a 60 min sobre una superficie típica de 100 a 300 km$^{2}$, mientras que los sistemas de tormentas grandes pueden producir más de un CG cada segundo, en áreas cien veces más grandes \cite{DwyerUman2014}.

El desarrollo de una descarga -CG \cite{DwyerUman2014} comienza con el movimiento de electrones libres hacia la región de carga positiva más baja de la nube, neutralizando una fracción significativa de su pequeña carga positiva, y continuando su viaje hacia la tierra. El mecanismo físico para mover la carga negativa a la Tierra es una descarga eléctrica llamada \textit{líder escalonado}, que se propaga en pequeños pasos de unos 10 m de longitud como se observa en el intervalo de tiempo $t=[\text{1.0},\, \text{19.00}]$ ms de la figura \ref{fig:cg_development}. Cada paso produce un pulso de luz visible, un pulso de energía de radiofrecuencia y un pulso de rayos X, en el rango de 200 keV. El tiempo entre los pasos luminosos es de $\sim 50\,\mu$s cuando el líder escalonado está muy por encima del suelo y de $\sim 10\,\mu$s cuando está cerca del suelo. En $t=$ 20.00 ms la carga negativa del l\'ider induce una carga positiva concentrada en la Tierra, o en objetos sobre la superficie como los para-rayos. Cuando la intensidad del campo eléctrico cerca de estas cargas se hace suficientemente grande, se inician descargas eléctricas ascendentes cargadas positivamente desde el suelo. Una de estas descargas ascendentes entra en contacto con una rama del líder que se mueve hacia abajo ($t=$ 20.10 ms), determinando así el punto de impacto del rayo y el canal principal de la corriente entre la nube y la tierra que se observa con gran luminosidad ($t=$ 20.20 ms). La luminosidad y la corriente del canal, en un proceso llamado el \textit{primer impacto de retorno}, se propaga continuamente hacia arriba del canal a una velocidad típicamente entre un tercio y la mitad de la velocidad de la luz. El impacto de retorno calienta el aire, que se expande y produce los truenos que generalmente escuchamos. 

Después de que la primera corriente del rayo ha dejado de fluir, el proceso puede terminar y a la descarga se le denomina \textit{de un solo impacto}. Sin embargo, en las regiones templadas el 80\% de los rayos -CG contienen más de un impacto, normalmente de tres a cinco \cite{DwyerUman2014}. Los impactos subsiguientes suelen estar separados por 40 o 50 ms y se inician solo si existe una carga negativa adicional cerca del canal anterior, en un tiempo inferior a 100 ms desde el cese de la corriente del primer impacto. A diferencia del líder escalonado el l\'ider de esta carga, denominado \textit{líder de dardo}, se propaga continuamente por el canal como se ilustra en $t=$ 60.00 ms y $t=$ 61.00 ms, para generar un segundo impacto ($t=$ 62.05 ms). Las corrientes del primer impacto son típicamente de 30 kA y las posteriores varían entre 10 - 15 kA \cite{DwyerUman2014}.

El 90\% de los rayos CG son iniciados por un l\'ider de carga negativa propagándose hacia abajo  (figura (a) de \ref{fig:CG4}), y el 10\% por un l\'ider de carga positiva (+CG) como se muestra en la figura (c) de \ref{fig:CG4}. Los otros dos tipos, que van realmente de la tierra a la nube (figuras (b) y (d) de \ref{fig:CG4}), son muy poco comunes y son iniciados hacia arriba desde las cimas de las montañas, torres altas hechas por el hombre, u otros objetos altos, hacia las regiones de carga de nubes \cite{DwyerUman2014}.

\begin{figure}
    \centering
    \includegraphics[scale=0.5]{figures/CG4.png}
    \caption[]{Cuatro tipos de descargas de nube a tierra: con un l\'ider de carga negativa (a) hacia abajo y (d) hacia arriba; y con un l\'ider de carga positiva (c) hacia abajo y (b) hacia arriba \cite{DwyerUman2014}. }
    \label{fig:CG4}
\end{figure}
%%%%%%%%%%%%%%%%%%%%%%%%%%%%
\section{El problema de iniciación de los rayos}
El problema de iniciación de los rayos dentro de las nubes, es uno de los mayores misterios de las Ciencias Atmosféricas y la Física de los rayos. El problema radica en que a partir de décadas de mediciones de campo eléctrico, realizadas directamente en el interior de las nubes, no se han logrado encontrar fuerzas de campo eléctrico lo suficientemente grandes como para producir una chispa. Sin embargo, rutinariamente se observan chispas muy grandes que se producen dentro de las tormentas eléctricas en forma de rayos. Esto sugiere que hay algo mal en las mediciones, o en la comprensión de cómo ocurren las descargas eléctricas en el entorno de una tormenta \cite{DwyerUman2014}. Para que la ruptura dieléctrica ocurra dentro de las tormentas se requiere un campo eléctrico de unos 23 kV/cm \cite{GurevichEtal2009}, pero el campo eléctrico medido y simulado es un orden de magnitud menor a este valor \cite{MarshallEtal2005, DwyerEtal2006, StolzenburgEtal2007}. Se cree que los Rayos Cósmicos (RCs) pueden estar involucrados en la formación y las tasas de cubrimiento de nubes en la Tierra, sin embargo, los resultados de estudios realizados en este campo no son suficientes para justificar los mecanismos físicos propuestos para describir el vínculo entre los RCs, los rayos y el clima \cite{KumarEtal2018}.

Los RCs son partículas de alta energía procedentes del espacio exterior, que al interactuar con la atmósfera cambian la concentración de iones atmosféricos, a través de la modulación del flujo de corriente en el circuito global eléctrico. La ionización en la troposfera y en la estratosfera se produce principalmente por los RCs. Mientras tanto, la radiación solar ultra-violeta (con longitud de onda de 10 a 120 nm) calienta e ioniza la alta atmósfera creando la ionosfera. Las variaciones de esta radiación debido al ciclo solar, produce variaciones en la ionosfera y en la alta atmósfera.   

Los RCs podrían causar cambios en el clima, ya que, afectan el campo el\'ectrico, la corriente, la conductividad en la atmósfera, las tasas de núcleos de condensación de las nubes, las descargas de rayos, la formaci\'on de hielo en ciclones, entre otras \cite{KumarEtal2018}. Svensmark et al. \cite{SvensmarkETal1997} y Marsh \cite{MarshSvensmark2000} reportaron una correlación entre los RCs y nubes de baja altitud ($<3$ km) alrededor del mínimo de RCs de 1990. Por otra parte, Kumar \cite{Kumar2017} demostró que la cobertura nubosa de bajo nivel, con una presión $> 680$ hPa, está positivamente correlacionada con el flujo de Rayos Cósmicos galácticos, utilizando mediciones en la Antártida en condiciones de buen tiempo. La correlación máxima es del 36\% durante el mínimo solar largo entre los años 2007-2009. Durante este periodo la ca\'ida de nieve increment\'o en un 14\% \cite{KumarEtal2018}. 

%%%%%%%%%%%%%%%%%%%%%%%%%%%%%%%%%%%%%%%%%%%%%%%%%%%%
\section{Distribución global y monitoreo de los rayos}
La tormentas en la Tierra se producen mayormente dentro de las regiones tropicales, es decir, entre $\pm 30^{\circ}$ de latitud respecto al ecuador, debido al máximo calentamiento solar que ocurre en los trópicos y a las células de circulación atmosférica general de Hadley, donde el calor es transportado en un movimiento celular con el aire ascendiendo por convección y desplazándose hacia las latitudes superiores por las capas altas de la atmósfera. El efecto integrado de las tormentas eléctricas globales y otras nubes electrificadas se puede estudiar a través del Circuito Eléctrico Global Atmosférico (CEGA), que se define como el curso continuo de la electricidad atmosférica entre la ionosfera (a 80 km de altitud) y la superficie terrestre. Se estima que las tormentas globales cargan negativamente la superficie de la Tierra con un valor medio de 500 kC y el potencial eléctrico medio generado entre estas capas es de 250 kV \cite{FullekrugEtal2006}. 

La actividad global de los rayos, se concentra más sobre las regiones continentales (las Américas, África y el sureste de Asia) que sobre los océanos, debido a que el calentamiento diario sobre la superficie de la tierra es mayor que en la superficie de los océanos \cite{FullekrugEtal2006}. La detección de rayos se basa en sistemas de observación óptica desde el espacio, como el OTD (Optical Transient Detector) \cite{ChristianEtal2003} o el LIS (Lightning Imaging Sensor) \cite{Christian1992}; y en la detección del ruido de radio (en la banda VLF/LF) generado por el rayo con arreglos de múltiples antenas para la reconstrucción del tiempo de llegada, como la red NLDP \cite{CumminsEtal1998}, el sistema EUCLID \cite{SchulzEtal2016} o la red mundial WWLLN \cite{RodgerEtal2004}.

En climatología, la base de datos de la HRAC (High Resolution Annual Climatology) \cite{AlbrechtEtal2016} es la representación más precisa hasta la fecha del total de rayos en todo el planeta, que contiene las observaciones realizadas por dos satélites de la NASA: el OTD y el LIS. El OTD fue el primer instrumento diseñado específicamente para detectar rayos desde el espacio durante el día y la noche. Fu\'e un prototipo del LIS, que se lanzó en 1997 y recompiló información hasta abril del 2015. A partir de estos años de datos se reconfirmó que la tasa media global de rayos es de 46/s, variando desde 35/s en el verano austral hasta 60/s en el verano boreal \cite{AlbrechtEtal2016}. Además se establecieron los sitios más activos del planeta en producción de rayos y los 10 primeros se muestran en la tabla \ref{topten}.

\begin{table}[]
\caption{Sitios con mayor Densidad de la Tasa de Rayos (FRD) a nivel mundial, tomado de \cite{AlbrechtEtal2016}}
\label{topten}
\begin{tabular}{|c|c|c|c|c|}
\hline
\textbf{Puesto} & \textbf{FRD (fl km$^{-2}$año$^{-1}$)} & \textbf{Lat ($^{\circ}$)} & \textbf{Lon ($^{\circ}$)} & \textbf{Sitio}               \\ \hline
1                       & 232.52       & 9.75             & -71.65            & Lago de Maracaibo, Venezuela \\ \hline
2                       & 205.31       & -1.85            & 27.75             & Kabare, R.D. del Congo       \\ \hline
3                       & 176.71       & -3.05            & 27.65             & Kampene, R.D. del Congo      \\ \hline
4                       & 172.29       & 7.55             & -75.35            & Cáceres, Colombia            \\ \hline
5                       & 143.21       & -0.95            & 27.95             & Sake, R.D. del Congo         \\ \hline
6                       & 143.11       & 34.45            & 72.35             & Daggar, Pakistan             \\ \hline
7                       & 138.61       & 8.85             & -73.05            & El Tarra, Colombia           \\ \hline
8                       & 129.58       & 5.25             & 9.35              & Nguti, Cameroon              \\ \hline
9                       & 129.50       & 0.25             & 28.45             & Butembo, R.D. del Congo      \\ \hline
10                      & 127.52       & -1.55            & 20.95             & Boende, R.D. del Congo       \\ \hline
\end{tabular}
\end{table}

%En la figura se muestran los top 10 FRD (flash rate density) \cite{AlbrechtEtal2016}. 
%\begin{figure}
 %   \centering
  %  \includegraphics[scale=0.7]{figures/hotspots.png}
  %  \caption{Caption}
   % \label{fig:my_label}
%\end{figure}



%%%%%%%%%%%%%%%%%%%%%%
%%%%% CHAPTER 2  %%%%%
%%%%%%%%%%%%%%%%%%%%%%

\chapter{Fenómenos Luminosos Transitorios en la alta atmósfera}\label{fenomenos}
Los Eventos Luminosos Transitorios, denominados como TLEs, son emisiones cortas de luz que ocurren encima de las tormentas eléctricas, en las capas más altas de la atmósfera, es decir, en la estratosfera, la mesosfera y en la parte baja de la ionosfera. Estos eventos se originan a partir de las descargas eléctricas que ocurren entre las nubes, o entre las nubes y el suelo \cite{DwyerUman2014}. Hasta ahora, se han reportado los siguientes tipos de TLEs: Sprites, Blue Starters, Blue Jets, Jets Gigantes, ELVES y Halos. La producci\'on de estos fenómenos electromagnéticos, que involucra corrientes eléctricas, ondas electromagnéticas y gases ionizados, aún no se entiende por completo. Por otro parte, varias observaciones indican la producción de Destellos de Rayos Gamma Terrestres (TGFs) por las tormentas eléctricas, y las condiciones en las que se generan sigue siendo una pregunta abierta.

Los TLEs se empezaron a reportar hace más de 130 años, cuando MacKenzie y Toynbee describieron en 1886 \cite{FullekrugEtal2006} lo que hoy sería conocido como un \textit{Jet Gigante}. Las primeras imágenes de un Sprite fueron capturadas cuando el profesor J.R. Winckler y sus estudiantes estaban probando un \textit{low-light television} (LLTV) para un próximo vuelo con cohetes de investigación en 1990 \cite{FranzEtal1990}. El video reveló un \textit{rayo de calor} distante, y columnas gemelas brillantes de luz que se extendían decenas de kilómetros en la atmósfera. Este tipo de rayo ascendente llamó la atención del programa de vuelos espaciales tripulados de los Estados Unidos, cuyas naves habían tenido varios encuentros desafortunados con rayos \cite{UmanRakov2003}, por lo que se instauró un programa de mapeo de rayos a meso-escala, introduciendo cámaras similares a los LLTV en el Transbordador Espacial. En este caso los vídeos revelaron un brillo transitorio que se iba ensanchando --seguramente un ELVES -- y más de una docena de rayos ascendentes sobre la tormenta. A partir de 1993 la NASA comenzó a financiar investigaciones terrestres y aéreas en este campo. En ese período se instaló el mismo LLTV de Winclker en la Estación de Campo Yucca Ridge de la FMA, donde se detectaron alrededor de 250 eventos durante varias horas por encima de una tormenta ubicada a 400 km \cite{Lyons1994A, Lyons1994B}. En 1994, grupos de la Universidad de Alaska y de la FMA notaron que estos eventos tenían asociadas señales características de audio de muy baja frecuencia (VLF). Ese mismo año empezó la campaña aérea que produjo las primeras imágenes a color de los Sprites y aparecieron inesperadamente los \textit{Blue Jets} \cite{SentmanEtal1995}. A partir de la campaña YRFS que se llevó a cabo entre 1994 y 1995 \cite{Lyons1996B}, se hizo evidente la correlación entre los Sprites y los rayos tipo +CG. De los 10.000 eventos confirmados sólo algunos Sprites fueron relacionados con rayos -CG \cite{BarringtonEtal1999}. 

\begin{figure}
    \centering
    \includegraphics[scale=0.75]{figures/tle_and_tgf.png}
    \caption[Fenómenos de la atmósfera superior generados por tormentas eléctricas]{Fenómenos de la atmósfera superior generados por tormentas eléctricas, incluyendo Destellos de Rayos Gamma Terrestres (TGFs) y Emisiones Luminosas Transitorias (TLEs). En esta figura se observan las dimensiones y las altitudes típicas en las que ocurren estos fenómenos. Las descargas eléctricas incluyen ``destellos azules'' en la parte superior de las tormentas eléctricas, Blue Jets, Jets Gigantes, Sprites, Halos, y ELVES (Tomado de \cite{Gaskill2018}).}
    \label{fig:tle_and_tgf}
\end{figure}

Por otra parte, los ELVES fueron confirmados ópticamente por primera vez en 1995 en la campaña YRFS \cite{FukunishiEtal1996}. En 1993 Taraneko \cite{TaranenkoEtal1993} había predicho este fenómeno, describiéndolo como un brillo intenso muy breve ($<1$ms) producido en la base de la ionosfera y asociado con el Pulso Electromagnético (EMP) de un rayo. Los programas ópticos de la YRFS y el Laboratorio Langmuir de Nuevo México, revelaron que lo que muchos de los primeros observadores pensaban que eran ELVES, en realidad se trataba del Halo que precede a algunos Sprites \cite{FullekrugEtal2006}. 

Para un estudio más fino de la estructura espacial y temporal de los Sprites, ELVES y Halos, se utilizan una variedad de sensores fotométricos, como el ``ojo de mosca", cámaras de alta velocidad, cámaras de alta resolución e imágenes telescópicas de alta velocidad \cite{FullekrugEtal2006}. A partir de estas observaciones se estimaron las características de los TLEs, como la altitud a la que se producen, su tiempo de vida, tamaño, forma, brillo y la tasa de producción en una tormenta típica. En la figura \ref{fig:tle_and_tgf} se muestran los diferentes fenómenos de la atmósfera superior producidos por tormentas eléctricas, incluyendo los TGFs y los TLEs. En ésta se puede observar la altitud característica de cada evento, así como el tamaño referencial de unos respecto a otros. En las secciones siguientes se describen las características más conocidas hasta ahora sobre estos fenómenos. 

\section{Sprites} 
Los Sprites son grandes ráfagas de luz tenue producidas encima de las tormentas eléctricas, que consisten en una cascada de plasma electrificado de color rojizo en su región superior, con ramificaciones azuladas en su parte inferior. El origen de sus filamentos eléctricos generalmente se produce alrededor de los 70 a 75 km de altitud \cite{FullekrugEtal2006}. Estas ramas altamente estructuradas, suelen propagarse primero hacia abajo, seguidas de una expansión hacia arriba en luminosidad, con un resplandor más difuso en la parte superior. Generalmente, los filamentos no se extienden más allá de los 40 km \cite{FullekrugEtal2006}. 

Todos los Sprites son visualmente diferentes, pero se pueden clasificar según su forma en: los tipo columna (c-Sprites) y los tipo zanahoria (z-Sprites). Los c-Sprites suelen ser muy estrechos, con un ancho del orden de 1 km. Son casi continuos, es decir, casi verticales, con filamentos que se extienden hacia abajo y hacia arriba. Estos pueden aparecer en grupos de una docena o más Sprites, repartidos en varias decenas de kilómetros. El clásico Sprite de ``zanahoria" tiene grupos de filamentos que se estrechan hacia abajo con elementos que se ensanchan hacia afuera en la parte superior \cite{FullekrugEtal2006}. Todavía se desconocen los mecanismos que producen estas formas, así como los detalles de su evolución. En la figura \ref{fig:sprite_evolution} se muestra la evolución temporal de un z-Sprite típico.

\begin{figure}[h!]
    \centering
    \includegraphics[scale=0.8]{figures/sprite_evolution2.png}
    \caption[Evolución temporal de un Sprite]{Imágenes del desarrollo de la estructura de un z-Sprite típico. Las etiquetas de tiempo se refieren al inicio del impacto de retorno del rayo \cite{CummerEtal2006}.}
    \label{fig:sprite_evolution}
\end{figure}

Los Sprites ocurren sólo después de las descargas de relámpagos CG con un líder positivo. Estos fenómenos pueden durar desde algunos milisegundos hasta decenas de milisegundos, tiempo suficiente para ser capturados por cámaras, o incluso por el ojo humano \cite{Maiorana2014}. Se estima que el brillo inherente de los Sprites está alrededor de 1 MR, con algunos casos raros donde han alcanzado desde 10 MR hasta 30 MR \cite{FullekrugEtal2006}. 

Las diferentes medidas realizadas en la superficie sugieren que, en una tormenta típica, se tiene un Sprite cada minuto. Un caso interesante son las Altas Llanuras de los Estados Unidos, donde después de la aparición inicial, los Sprites usualmente continúan saliendo por varias horas. En ocasiones, estas tormentas han producido de 400 a 750 Sprites en 4-5 horas \cite{FullekrugEtal2006}.

\section{Blue Starters, Blue Jets y Jets Gigantes}
Los Blue Starters y Blue Jets son rayos que salen del tope de las nubes hacia arriba, se extienden hasta unos 30 a 40 km de altitud y luego se desvanecen. Estos eventos tienen forma de cinta, son de color azul y suelen ser más brillantes que los Sprites ($>1$ MR). Adem\'as, se propagan con velocidades alrededor de $10^5$ m/s y su tiempo de vida ronda en los 300 ms \cite{DwyerUman2014}. 

Algunos Jets se denominan gigantes, puesto que pueden alcanzar hasta 70 km de altitud. La forma de un Jet Gigante es más compleja, como se observa en la figura \ref{fig:tle_and_tgf}, por lo que generalmente se describen como la mezcla de un Blue Jet y un Sprite, debido a que su parte baja es cónica como un Jet y luego se dispersa en un brillo difuso y tenue, de un tono rojizo como un Sprite. 

Los Blue Jets ocurren más o menos con la misma frecuencia que los Sprites, mientras que los Jets Gigantes son bastante raros, con una tasa de 13 en tres años \cite{chen2008}. Recientemente, investigadores de Estados Unidos y de la Universidad de Cataluña \cite{VanEtal2019}, capturaron con cámaras de alta velocidad el nacimiento y las etapas de evolución de Jets Gigantes en la costa norte colombiana. En la figura \ref{fig:gigantic_jets_colombia} se aprecia claramente la evolución temporal de un Jet Gigante con sus cuatro etapas: el Jet líder, el Jet completamente desarrollado, el Jet de seguimiento y el líder final. 

\begin{figure}
    \centering
    \includegraphics[scale=0.45]{figures/gigantic_jets_colombia.png}
    \caption[Jets Gigantes capturados en la costa norte de Colombia]{a) Cinco Jets Gigantes capturados en la costa norte de Colombia: Santa Marta (GJ 2-6), Barranquilla (GJ 7) y Cartagena (GJ 12). b) Etapas de la evolución de un Jet Gigante grabadas con una cámara de 2.3 mega-píxeles a 20 imágenes por segundo; se puede observar el Jet líder en los primeros 150 ms seguido del Jet completamente desarrollado, el Jet de seguimiento y el líder final \cite{VanEtal2019}.}
    \label{fig:gigantic_jets_colombia}
\end{figure}

\section{ELVES}
Un ELVES es un disco toroidal de luz roja de rápida expansión, que se genera en la parte baja de la ionosfera (la altitud está entre los 80-100 km). Típicamente son generados por el pulso electromagnético de una descarga -CG y su diámetro puede extenderse hasta más de 300 km. El EMP se propaga como una onda esférica desde la base del canal del rayo, intersectando la ionosfera inferior como un anillo que se expande más rápido que la velocidad de la luz en ese medio. El campo eléctrico del EMP produce la calefacción de electrones libres, aumentando las excitaciones inducidas por colisión, la ionización y las emisiones de luz óptica \cite{FullekrugEtal2006}.

Los ELVES están asociados a descargas -CG con corrientes de pico más altas que la de los rayos que producen los Sprites, es decir, mayor a 80 kA. A partir de los datos tomados desde la tierra, se tiene que los Sprites y los ELVES ocurren entremezclados en las mismas tormentas, aunque la proporción varía considerablemente de una tormenta a otra \cite{FullekrugEtal2006}. Por otra parte, las observaciones satelitales de la misi\'on espacial ISUAL (Imager of Sprites and Upper Atmospheric Lightning), sugieren que los ELVES son los TLEs que ocurren con m\'as frecuencia, con una tasa global de 3.23 eventos min$^{-1}$, superando en número a los Sprites (0.50 min$^{-1}$), a los Halos (0.39 min$^{-1}$) y a los Jets Gigantes (0.01 min$^{-1}$) \cite{chen2008}, como se observa en la figura \ref{fig:TLE_global_rate}. Además, los ELVES ocurren con más frecuencia sobre los océanos y su tasa aumenta cuando la temperatura de la superficie del mar excede los 26$^{\circ}$C. Esto puede ser debido a que los rayos con corrientes de pico altas ($>80$kA) son 10 veces más comunes sobre los océanos que sobre la tierra \cite{chen2008}. 

\begin{figure}
    \centering
    \includegraphics[scale=0.33]{figures/TLE_global_rate.png}
    \caption[Densidad de ocurrencia global de los TLEs m\'as comunes]{Densidad de ocurrencia global de los TLEs m\'as comunes: (a) Sprites y Jets Gigantes, (b) ELVES y (c) Halos. Se puede observar que los ELVES son m\'as frecuentes que los Sprites, seguidos de los Halos y los Jets Gigantes. (d) Se presenta la temperatura de la superficie del mar entre julio de 2004 y diciembre de 2005, para compararla con la densidad de ocurrencia de los ELVES (tomado de \cite{chen2008}).}
    \label{fig:TLE_global_rate}
\end{figure}

El nombre ELVES era originalmente un acrónimo de \textit{Emission of Light and Very low frequency perturbations due to Electromagnetic pulse Sources}, aunque ahora pocos autores mencionan este nombre completo. Estos fenómenos pueden ser tan brillantes como un Sprite típico ($\sim$1MR) pero su duración, de algunas decenas de microsegundos, hace que sean imperceptibles para el ojo humano y difíciles de detectar usando sistemas de vídeo. Es por esto que generalmente se detectan a partir de arreglos de fotómetros, con resoluciones temporales alrededor de los 0.3 ms \cite{Maiorana2014}. En la figura \ref{fig:elves_photo} se muestra una foto impresionante de un ELVES acompañado por tres Sprites, tomada por el astrónomo Timo Kantola que corrió con la suerte de capturar esta escena. En la sección \ref{deteccion} se resume cómo los grupos más resaltantes logran la detección de ELVES con sus respectivos instrumentos. 
\begin{figure}
    \centering
    \includegraphics[scale=0.65]{figures/elves_photo.png}
    \caption[Foto de un ELVES tomada por el astrónomo Timo Kantola]{Foto de un ELVES junto a tres Sprites tomada por el astrónomo Timo Kantola, en Pieksämäki Finlandia.}
    \label{fig:elves_photo}
\end{figure} 

\section{Halos}
Son emisiones de luz difusa en forma de disco, que ocurren dentro de un campo eléctrico casi-estático como los Sprites. Al principio estos discos se confundían con ELVES y como aparecían junto a los Sprites solían llamarlos \textit{Sprelves}. Ahora se sabe que los Halos pueden aparecer junto con los Sprites o por sí solos. De hecho, en una campaña de sensores ópticos en globos \cite{BeringEtal2004}, se observó que muchos Halos asociados a rayos +CG iban acompañados de Sprites, a diferencia de los Halos asociados a rayos -CG.

Además, su diámetro es mucho menor que el de los ELVES --máximo de 100 km-- y su duración es mayor (alrededor de 1 ms). Estos eventos se producen por debajo de los ELVES, entre los 65 y 80 km de altitud. Vídeos de alta velocidad \cite{ArmstrongLyons2000, StanleyEtal1999, StenbaekEtal2000} han demostrado que el Halo es un resplandor amorfo descendente en forma de lente que se inicia típicamente uno o dos milisegundos después del impacto de retorno, y persiste de uno a tres milisegundos. Su brillo es rojo y ronda entre los 0.5 y 1 MR \cite{FullekrugEtal2006}.

\section{Destellos de Rayos Gamma Terrestres}
Los Destellos de Rayos Gamma Terrestres son emisiones transitorias breves, que duran menos de unos pocos milisegundos y los fotones emitidos alcanzan energ\'as de decenas de MeV \cite{neubertEtal2020}. Los TGFs son producidos por la radiación de Bremsstrahlung de electrones acelerados en los campos eléctricos de las tormentas, alcanzando energías relativistas en el régimen de escape libre. Los electrones en este régimen pueden ser liberados por las interacciones de los Rayos Cósmicos con la atmósfera, o de electrones térmicos acelerados en los campos de las puntas de los rayos \cite{neubertEtal2020}. Este fenómeno ha sido observado por sat\'elites que pasan sobre las tormentas, como el CGRO \cite{FishmanEtal1994}, el RHESSI \cite{Smith2005Etal}, m\'as recientemente el sat\'elite AGILE \cite{MarisaldiEtal2010}, el Fermi GBM \cite{RobertsEtal2018} y el ASIM \cite{neubertEtal2019} a bordo de la Estaci\'on Espacial Internacional (ISS).

Inicialmente se pensaba que los TGFs se originaban de los TLEs de gran altitud, pero más tarde se observó que su fuente estaba dentro de las nubes de la tormenta. Estudios recientes sobre la estrecha asociación entre la producción de TGFs y los procesos de rayos simultáneos, han demostrado que \'estos se producen durante el líder inicial de las descargas IC y que algunas emisiones de radio de baja frecuencia (LF) pueden provenir directamente del propio TGF \cite{LyuEtal2018}. De hecho, ya se han identificado varias emisiones de radio de rayos asociados a TGFs, con observaciones que enmarcan el primer escenario claro de los procesos del rayo que ocurren antes, durante y después de la producción del TGF. En este caso, todos los TGFs se produjeron varios milisegundos después de que el líder se iniciara y cuando los líderes alcanzaron de 1 a 2 km de longitud \cite{CummerEtal2015}. Recientemente, el ASIM observ\'o simultáneamente un TGF y emisiones UV de un ELVES. El escenario propuesto es el siguiente: el TGF se produce en la etapa inicial del rayo cuyo pulso de corriente genera un ELVES. Las mediciones sugieren que el inicio de la corriente es rápido y de gran amplitud, un prerrequisito para producir los ELVES, y que el TGF se genera en los campos eléctricos asociados al líder del rayo. Esto sugiere que, después de todo, existe una conexión entre los TLEs y los TGFs \cite{neubertEtal2020}. 


%%%%%%%%%%%%%%%%%%%%%%
%%%%% CHAPTER 3  %%%%%
%%%%%%%%%%%%%%%%%%%%%%
\chapter{Detección global de los ELVES}\label{deteccion}
La primera observación clara de un ELVES se hizo a partir de una cámara de alta velocidad a bordo del Transbordador Espacial en su misión STS-41 (1990) \cite{BoeckEtal1992}. Estos dispositivos son lo suficientemente sensibles para detectar las emisiones de luz en el aire, puesto que una simple molécula de nitrógeno ionizada emite luz a 427nm y 391nm, y la cámara es sensible a longitudes de onda entre los 360 y 720 nm, con un pico de sensibilidad en 440 nm. En este caso, el vídeo fue grabado en formato blanco y negro, con una resolución temporal de $\approx 17$ ms. La imagen fue mejorada restando el fondo y expandiendo la escala de grises, para determinar si el aumento en el brillo era una fluctuación aleatoria o la luminosidad de un rayo. Este ELVES fue observado a una altitud alrededor de los 95 km, en coincidencia con un rayo producido en una tormenta oceánica directamente debajo \'este  \cite{BoeckEtal1992}.  

Más adelante, estos fenómenos se empezaron a estudiar a partir de arreglos lineales de fotómetros, con un tiempo de resolución mejorado a $40\,\mu$s. El campo de visión individual de cada uno de los fotómetros dispuestos en un arreglo horizontal \cite{InanEtal1997}, originaron las primeras medidas de la expansión lateral rápida de la luminosidad óptica de los ELVES, con una resolución temporal de $30\,\mu$s . 

Por otra parte, el experimento ISUAL \cite{chen2008} --a bordo del sat\'elite FORMOSAT-2-- se diseñó para detectar diferentes TLEs a través de foto-multiplicadores (PMT) convencionales y multi-ánodos, con resoluciones temporales de 100 y $50\,\mu$s respectivamente. Los datos registrados por ISUAL desde julio de 2004 a junio de 2007 (5.434 ELVES, 633 Sprites, 657 Halos y 13 Jets Gigantes) permitieron concluir que los ELVES ocurren con m\'as frecuencia respecto a los demás TLEs, como se observa en la figura \ref{fig:TLE_global_rate}. Otras misiones espaciales que incluyen observaciones de ELVES en sus programas son TARANIS \cite{lefeuvre2008taranis} y Firefly \cite{rowland2011nsf}. TARANIS se enfoca en estudiar las regiones donde se originan los TLEs y TGFs, sus mecanismos de generación, los fenómenos asociados, así como el estudio de los parámetros de entrada para modelar la variación de la atmósfera y el circuito eléctrico global. Por su parte, Firefly emplea nano-satélites diseñados para explorar la relación entre los rayos y los TGFs, para descifrar qué tipos de rayos producen estos haces de electrones y los TGFs asociados.

El Monitor de Interacciones Atmósfera-Espacio (ASIM) \cite{neubertEtal2019} a bordo de la Estaci\'on Espacial Internacional, comprende la primera misión espacial con un conjunto completo de instrumentos diseñados para medir rayos, TLEs y TGFs. ASIM se desarroll\'o en el marco de la Agencia Espacial Europea (ESA) y fu\'e lanzado en abril de 2018, contando con un monitor de rayos X y rayos gamma que mide fotones de 15 keV a 20 MeV, y un conjunto de tres fotómetros y dos cámaras que miden en las bandas de 180-250 nm, 337 nm y 777.4 nm. Sus objetivos fundamentales son el conteo mundial exhaustivo de TLEs y TGFs abarcando todas las horas nocturnas y estaciones locales; toma de datos para comprender los procesos cinéticos fundamentales de los TLEs y TGFs; y entender la relación de los TLEs y los TGFs con la actividad de los rayos.

%Otros objetivos adicionales que pueden abordarse con los instrumentos se relacionan con la física espacial, como las auroras y los meteoros, y con la observación de la Tierra, como los efectos del polvo y el aerosol en la electrificación de las nubes. 

Mientras que estas misiones fueron desarrolladas específicamente para el estudio de rayos, TGFs y TLEs, el Detector de Fluorescencia (FD) del Observatorio Pierre Auger descubrió por accidente un candidato a ELVES durante un turno de observación en el año 2005 \cite{Mussa2019}. Este observatorio, ubicado en Malargue, Argentina, es el más grande del mundo en infraestructura para el estudio de Rayos Cósmicos de ultra-alta energía, que combina las facilidades de un Detector de Superficie (SD) y un Detector de Fluorescencia (FD). Además de su actividad principal, el observatorio inició un programa de estudios en Cosmo-geofísica, para explotar las características de su FD en el estudio de eventos luminosos transitorios. El Observatorio Pierre Auger podría ser el mejor instrumento con bases en la tierra para la observación de eventos en la banda del ultravioleta cercano \cite{MussaCiaccio2012}, por lo que a continuación se describe su funcionamiento.  

%%%%%%%%%%%%%%%%%%%%%%%%%%%%%%%%%%%
%%%%%%%%%%%%%%%%%%%%%%%%%%%%%%%%%%%
\section{ELVES en el Observatorio Pierre Auger}
\begin{figure}
    \centering
    \includegraphics[scale=0.5]{figures/auger_sites.png}
    \caption[Distribuci\'on de los Detectores de Superficie y Fluorescencia del Observatorio Pierre Auger.]{Distribuci\'on de los Detectores de Superficie y de Fluorescencia del Observatorio Pierre Auger. Los puntos grises muestran las posiciones de las estaciones del SD, mientras que las l\'ineas indican el campo de visi\'on de los 24 telescopios del FD. Los telescopios est\'an ubicados en el per\'imetro del SD en cuatro edificios: Los Leones, Los Morados, Loma Amarilla y Coihueco \cite{AbrahamEtal2010}. }
    \label{fig:auger_sites}
\end{figure}
El detector de Fluorescencia del Observatorio Pierre Auger consta de cuatro sitios de observaci\'on localizados en la cima de pequeñas colinas (Los Leones, Los Morados, Loma Amarilla y Coihueco), en los límites del arreglo del SD (ver figura \ref{fig:auger_sites}). Los cuatro edificios del FD contienen seis telescopios independientes, cada uno con un campo de visión de 30 grados en acimut $\times$ 30 grados de elevación (ver figura superior de \ref{fig:fd_scheme}). La combinación de los seis telescopios cubre 180 grados en acimut \cite{AbrahamEtal2010}. En cada telescopio, la luz entra a través de una ventana con un filtro UV y es enfocada a través de un espejo esférico segmentado sobre la cámara. Los elementos del telescopio se muestran en la parte inferior izquierda de la figura \ref{fig:fd_scheme}. Las c\'amaras de los telescopios est\'an compuestas de 440 PMTs hexagonales en un arreglo matricial de 22 filas por 20 columnas como se muestra en la figura c de \ref{fig:fd_scheme}. Estos PMTs representan los p\'ixeles de la c\'amara y est\'an ubicados sobre una superficie esf\'erica con pasos de 1.5$^\circ$ \cite{AbrahamEtal2010} (ver figura a y b de \ref{fig:fd_scheme}). 

La longitud de onda de la luz detectada en cada PMT varia desde 300 hasta 420 nm y los pulsos de luz se digitalizan cada 100 ns. Para la selección en línea de los eventos, la data de cada PMT es procesada a través de un sistema de disparo de tres etapas \cite{MussaCiaccio2012}. Los 24 telescopios del FD son ideales para la observación de los ELVES, pero para la selecci\'on en línea de estos eventos con una eficiencia razonable, se han realizado estudios basados en sub-triggers auxiliares que permitieron modificar el tercer nivel del sistema de disparo \cite{Mussa2019}.

\begin{figure}
    \centering
    \includegraphics[scale=0.5]{figures/eye_scheme.png}
    
    \includegraphics[scale=0.35]{figures/telescope_scheme.png} \,\,\, \includegraphics[scale=0.35]{figures/camera.png}
    \caption[]{Arriba: estructura de los edificios del FD con seis telescopios de fluorescencia independientes. Cada uno tiene un campo de visión de 30 $\times$ 30 grados en acimut y elevación, que en total cubren 180 grados en acimut. Abajo, a la izquierda: estructura y componentes de los telescopios de cada FD; a la derecha: construcci\'on geom\'etrica de las c\'amaras del FD: a) los centros de los píxeles est\'an ubicados sobre una superficie esf\'erica en pasos de 1.5 grados, b) posicionamiento de los vértices del píxel respecto a su centro, y c) arreglo de los 440 píxeles de la c\'amara en una matriz de 22 $\times$ 20. Tomado de \cite{AbrahamEtal2010}.}
    \label{fig:fd_scheme}
\end{figure}

%%%%%%%%%%%%%%%%%%%%%%%%%%%%%%%%%%%
\subsection{Sistema de disparo del FD}
El disparador del FD está estructurado en tres niveles \cite{Mussa2019}: 
\begin{enumerate}
    \item El disparo del primer nivel (DPN) opera a nivel del píxel, con un umbral ajustable que mantiene la tasa de activación de los PMT a 100 Hz.
    \item El disparo del segundo nivel (DSN) busca segmentos con trayectorias de al menos 5 píxeles adyacentes que pasaron la primera etapa. 
    \item El disparo del tercer nivel (DTN) está diseñado para excluir eficientemente los rayos cercanos. Los eventos de rayos presentan una alta multiplicidad de píxeles disparados que se distribuyen de forma aleatoriamente temporal a través de la cámara, dentro de una ventana de tiempo de 100 $\mu$s. 
\end{enumerate}

El DTN basado en la multiplicidad, instalado a finales del 2007 como reemplazo de una versi\'on previa menos eficiente, permiti\'o el hallazgo inesperado de ELVES en los primeros datos del Observatorio Pierre Auger \cite{MussaCiaccio2012}). Después de un estudio de cuatro años de eventos de DPN, el DTN se modificó y arrojó 58 nuevos candidatos de ELVES \cite{Mussa2019}.

El nuevo DTN verifica la evoluci\'on angular del frente de luz alrededor del primer p\'ixel disparado. Para un conjunto de p\'ixeles de la misma columna que \'este, el algoritmo requiere que al menos dos p\'ixeles antes y dos p\'ixeles desp\'ues del central tengan un pulso. Adem\'as, 80\% de \'estos deben mostrar un incremento en el tiempo del pulso. En comparación con los Rayos Cósmicos, los ELVES depositan una gran cantidad de luz, por lo que es necesario al menos un píxel con un pulso de amplitud mayor que 50 cuentas ADC \cite{Mussa2019}. En la figura \ref{fig:crs_elves} se puede observar un Rayo C\'osmico registrado en la c\'amara como una l\'inea de p\'ixeles activados con un patr\'on geom\'etrico y una clara secuencia temporal, mientras que el ELVES se ve como un frente de onda con una evolución radial. 

\begin{figure} 

    \centering
    \includegraphics[scale=0.6]{figures/cr_vs_elves.png}
    \caption[Observaci\'on de Rayos C\'osmicos y ELVES en un telescopio del FD del Observatorio Pierre Auger]{Observaci\'on de Rayos C\'osmicos y ELVES en un telescopio del FD. Tiempo de propagaci\'on proyectado sobre la c\'amara
    de A) un Rayo C\'osmico y B) un ELVES. El tiempo se mide a partir del primer disparo del sistema, siendo los p\'ixeles azules los primeros en activarse y los rojos los \'ultimos. Se puede observar que el Rayo C\'osmico activa una l\'inea de p\'ixeles con una clara secuencia temporal y que el ELVES se ve como un frente de onda con una evolución radial.}
    \label{fig:crs_elves}
\end{figure}

Desde el 2013, el Observatorio Pierre Auger ha estado tomando datos con este sistema de disparo para el estudio de ELVES. Durante el primer año se observ\'o que las trazas estándar del FD, de 72 $\mu$s de largo, no permit\'ian ver la luz emitida desde la regi\'on de la ionosfera localizada verticalmente sobre el rayo fuente. Entonces, se modific\'o el esquema de lectura del FD (denominado \textit{Extended Readout}) para adquirir tres cuadros consecutivos en estos casos especiales. Este cambio permite el estudio de la distribuci\'on angular de la emisi\'on de luz por encima del rayo. Los datos de ELVES desde 2014 hasta 2016 se tomaron con una traza m\'axima de 300 $\mu$s de longitud. A partir del 2017, esta traza se extendi\'o hasta 900 $\mu$s para observar completamente la regi\'on de alta intensidad de luz en la mayor\'ia de los eventos \cite{Mussa2019}. Hasta septiembre del año 2018 se han registrado m\'as de 4000 ELVES en el Observatorio Pierre Auger, como se muestra en el cuadro \ref{eventos_elves}, donde los observados en un solo sitio del FD se denominan Simples, en dos sitios son los Estéreo y en tres sitios los Tripletes.  

\begin{table}[]\centering
\caption{N\'umero de ELVES registrados en el FD desde el 2013 hasta el 2018 \cite{Mussa2019}}
\label{eventos_elves}
\begin{tabular}{lcccc}
\rowcolor[HTML]{EFEFEF} 
Año         & Simple & Estéreo & Triplete & Total                        \\
2013 (4-12) & 214    & 83      & 8        & \cellcolor[HTML]{EFEFEF}305  \\
2014        & 425    & 128     & 19       & \cellcolor[HTML]{EFEFEF}572  \\
2015        & 686    & 117     & 11       & \cellcolor[HTML]{EFEFEF}814  \\
2016        & 673    & 151     & 21       & \cellcolor[HTML]{EFEFEF}845  \\
2017        & 906    & 297     & 52       & \cellcolor[HTML]{EFEFEF}1255 \\
2018 (1-9)  & 527    & 99      & 15       & \cellcolor[HTML]{EFEFEF}641  \\
\rowcolor[HTML]{EFEFEF} 
Total       & 3431   & 875     & 126      & 4432                        
\end{tabular}
\end{table}

%%%%%%%%%%%%%%%%%%%%%%%%%%%%%%%%%%%
\subsection{Reconstrucci\'on de la ubicaci\'on y la emisi\'on de luz de los ELVES }
La reconstrucción apunta a la localización precisa en tiempo y espacio del rayo que origin\'o el ELVES, y a la medición de la distribución angular de la emisión de luz desde la base de la ionosfera. Para esto, lo primero que se hace es un ajuste de una o m\'as gaussianas asim\'etricas a la traza ADC de cada uno de los píxeles. La reconstrucci'on de la ubicaci'on del rayo es un proceso de cuatro pasos \cite{Mussa2019}:
\begin{enumerate}
    \item Se realiza un conjunto de ajustes polinomiales en los tiempos $T_i$ del pulso de cada fila y cada columna, para obtener una primera estimaci\'on de la longitud y la latitud del rayo. 
    
    \item Se realiza otro ajuste para minimizar el $\chi ^{2}= \sum_{i=1}^{N} \left( T_i -\Delta T_0 -\overline{OPS}_i/c \right)^{2}$, donde $\Delta T_0$ es el tiempo entre la emisión del EMP de la fuente y la observación de la primera luz en el diafragma del FD, $\overline{OPS}_i$ es la suma de las distancias $\overline{\text{SP}}$ y $\overline{\text{PO}}$ (ver figura \ref{fig:emission_layer}). En este caso se asume que la capa de emisión no tiene grosor y que la altitud es $H_{EM}=92$ km.
    
    \item El tercer ajuste se hace dejando libre la altitud de emisión $H_{EM}$.
    
    \item El último ajuste se realiza permitiendo también que la altitud del rayo $H_B$, varíe sobre el nivel del mar.
\end{enumerate}
\begin{figure}
    \centering
    \includegraphics[scale=0.3]{figures/emission_layer.png}
    \caption[Geometr\'ia de reconstrucci\'on de los ELVES en el Observatorio Pierre Auger]{Geometr\'ia de reconstrucci\'on de los ELVES en el Observatorio Pierre Auger \cite{Mussa2019}. $\overline{\text{SP}}$ es la distancia de la fuente del EMP (S) a la capa de emisi\'on donde se produce el ELVES y $\overline{\text{PO}}$ es la distancia desde la capa de emisi\'on hasta el FD del observatorio. En el proceso de reconstrucci\'on se asume que la capa de emisi\'on no tiene grosor \cite{Mussa2019}.}
    \label{fig:emission_layer}
\end{figure}

Después de haber localizado el rayo fuente, se procede a calcular la cantidad total de fotones irradiados por el ELVES. Para esto es necesario transformar la cantidad de luz observada en cada píxel, $P_i(FD)$, a la densidad superficial de fotones ,$\Phi_i$, en la base de la ionosfera:
\begin{equation}
    \Phi_i=P_i(FD)\times C_i(geom)\times C_i(atmo).
\end{equation}
La corrección geométrica, $C_i(geom)$, tiene en cuenta el ángulo sólido entre el diafragma del FD y el tamaño de la superficie emisora en la base de la ionosfera ($A_{emis}$):
\begin{equation}
    C_i(geom)=\frac{\overline{OP}^{2}}{A_{mirror}A_{emis}}.
\end{equation}

Por otra parte, la corrección atmosférica incluye las profundidades ópticas, $\tau$, de los aerosoles y las moléculas a través del camino de la luz:
\begin{equation}
    C_i(atmo)= \exp\left(-(\tau_{OP, mol}+\tau_{OP, aer})\times AM(\theta)\right),
\end{equation}

donde $AM(\theta)$ es la masa de aire. 

La corrección geométrica y la atmosférica dependen fuertemente del número de fila y débilmente del número de columna de la cámara. Después de aplicar estas correcciones, en el caso de un dipolo perfecto, la distribución angular de la emisión es azimutalmente simétrica con respecto a la vertical del rayo, y la cantidad total de luz emitida desde la base de la ionosfera puede extrapolarse fácilmente a partir de la fracción observada. Sin embargo, se necesita más trabajo para modelar dipolos inclinados o patrones de emisi\'on m\'as complejos \cite{Mussa2019}.


%%%%%%%%%%%%%%%%%%%%%%%%%%%%%%%%%%%
\subsection{Clasificación de ELVES múltiples}
\begin{figure}
    \centering
    \includegraphics[scale=0.1]{figures/double_elves_1.png}\includegraphics[scale=0.7]{figures/categorias.png}
    \caption[Clasificaci\'on de los ELVES m\'ultiples]{Izquierda: ejemplo de un evento con m\'as de diez p\'ixeles donde sus trazas de 300$\mu$s muestran un doble pulso; en la ampliaci\'on se puede observar la diferencia temporal $\Delta T$ entre los dos pulsos. Derecha: diferencia temporal entre dos pulsos de una traza, $\Delta T$, versus el tiempo del primer pulso, $T_1$, en eventos de ELVES m\'ultiples clasificados en cuatro categor\'ias \cite{Mussa2019}.}
    \label{fig:categorias}
\end{figure}
En un evento, cuando m\'as de diez p\'ixeles muestran m\'as de un pulso en las trazas (ver figura izquierda de \ref{fig:categorias}) el ELVES se denomina m\'ultiple y se realiza un an\'alisis diferente al descrito anteriormente. El primer paso consiste en graficar la diferencia temporal entre los dos pulsos, $\Delta T$, respecto al tiempo del primer pulso, $T_1$, representado en la figura \ref{fig:categorias} donde estos eventos se pueden clasificar en cuatro categor\'ias \cite{Mussa2019}:

\begin{enumerate}
    \item Cuando $\Delta T$ es constante y muy pequeño ($<50 \,\mu$s) el doble pulso se puede explicar debido a un patrón de interferencia producido por un EMP emitido por un rayo IC, que rebota en el suelo. 
    \item En el caso donde $\Delta T \geq 50\mu$s y es constante, los eventos se pueden relacionar con las primeras etapas del rompimiento inicial de los rayos o con los impactos de retornos muy cercanos. 
    \item La tercera categor\'ia, donde $\Delta T$ es linealmente decreciente con el tiempo del pulso, se puede explicar como un ELVES simple superpuesto con alguna otra clase de luz transitoria, con un tiempo de subida mucho más largo.
    \item En la \'ultima categor\'ia se tienen cambios m\'as abruptos de $\Delta T$ pertenecientes a ELVES con m\'as de dos pulsos en las trazas. Este patr\'on podr\'ia estar asociado a la producci\'on de TGFs.
    \end{enumerate}

Estas suposiciones no est\'an del todo comprobadas, por lo que a\'un queda mucho por estudiar en cuanto a la din\'amica de los ELVES y la generaci\'on de los dos pulsos en sus trazas. Es necesario seguir profundizando en esta \'area de investigaci\'on para describir con mayor precisión estos fen\'omenos, a partir del estudio de las trazas de los EMP producidos por rayos CG o IC, los impactos de retorno muy cercanos, la altura de la nube, as\'i como la relaci\'on de los TGFs con los TLEs. 


%%%%%%%%%%%%%%%%%%%%%%%%%%
%\chapter*{Conclusiones}
%\addcontentsline{toc}{chapter}{Conclusiones}
%A pesar de que existen varios programas para el estudio de los TLEs, todav\'ia quedan muchas preguntas abiertas. Por ejemplo,  Los rayos El doble picos de los ELVES Los TGFs Este campo de investigaci\'on tiene mucho por explorar 

\bibliographystyle{unsrt}
\bibliography{monografia}

\end{document}